% Required files to compile:
% 1. automation_logo.pdf
% 2. beamerthemeJUB.sty
% 3. bibliography_file.bib
% 4. jjlogo.pdf
% 5. large-corner.pdf
% 6. small-corner.pdf

\documentclass{ctexbeamer}

% 这里更改风格
\usetheme{JUB}
% \usetheme{ALUF}

\usepackage[T1]{fontenc}
\usepackage[scaled]{helvet}
\usepackage{libertine}
% \usepackage{times}

\begin{document}

  \title{Main Title 主标题}
  % \subtitle{Subtitle 副标题} % optional
  \author{Author 管华}
  \date{\today}
  \institute{Renmin University of China\\ 中国人民大学\\ \url{http://www.ruc.edu.cn/}} % optional

  \begin{frame}[plain,t]
      \titlepage
  \end{frame} % ================================================================

  \begin{frame}{Table of Contents 目录}
      \tableofcontents
  \end{frame} % ================================================================

  \section{Introduction}
  \label{Sec:introduction}
  \begin{frame}{1. Introduction 引言}
      \framesubtitle{frame subtitle 页面副标题}
      This tex is a~\LaTeX\ Beamer template using the JUB Beamer theme~\cite{JUBTheme} produced by Billy Okal.

      \bigskip

      此文件是一个使用了Billy~ Okal制作的JUB Beam主题\cite{JUBTheme}的\LaTeX \ Beamer模板。

      \begin{figure}
          \begin{center}
              \includegraphics[scale=0.1]{latex.png}
          \end{center}
          \caption{\LaTeX \ beamer}
          \label{Fig:latex_beamer}
      \end{figure}
  \end{frame} % ================================================================

  \section{Mathematical model}
  \label{Sec:model}
  \begin{frame}{2. Mathematical model 数学模型}
      Lists 列表

      \begin{itemize}
          \item Apple
          \item Orange
          \item Banana
      \end{itemize}

      \begin{enumerate}
          \item Monday
          \item Tuesday
          \item Wednesday
      \end{enumerate}

      \begin{description}
          \item[Description list] is a type of list to describe items.
          \item[Description list] 是一种用于描述的列表。
      \end{description}
  \end{frame} % ================================================================

  \section{Empirical experiments}
  \label{Sec:experiments}
  \begin{frame}{3. Empirical experiments 实验}
      Blocks 区块

      \begin{definition}[Pythagorean theorem]
          The Pythagorean theorem is a fundamental relation in Euclidean geometry among the three sides of a right triangle.
      \end{definition}

      \begin{theorem}[Pythagorean theorem]
          $a^2 + b^2 = c^2$
      \end{theorem}
  \end{frame} % ================================================================

  \begin{frame}{3. Empirical experiments 实验}
      Blocks 区块

      \begin{exampleblock}{For example}
          $3^2 + 4^2 = 5^2$
      \end{exampleblock}

      \begin{alertblock}{Note}
          Note that the Pythagorean theorem can only be applied to right triangles.
      \end{alertblock}
  \end{frame} % ================================================================

  \begin{frame}
      \begin{center}
          \Huge{\bf{Thank you!}}
      \end{center}
  \end{frame} % ================================================================

  \bibliographystyle{IEEEtran} % comment this line if nothing is cited
  \bibliography{bibliography_file} % comment this line if nothing is cited

\end{document}
